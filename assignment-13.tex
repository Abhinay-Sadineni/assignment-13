\documentclass{beamer}
\usetheme{CambridgeUS}

\setbeamertemplate{caption}[numbered]{}

\usepackage{enumitem}
\usepackage{tfrupee}
\usepackage{amsmath}
\usepackage{amssymb}
\usepackage{gensymb}
\usepackage{graphicx}
\usepackage{txfonts}

\def\inputGnumericTable{}

\usepackage[latin1]{inputenc}                                 
\usepackage{color}                                            
\usepackage{array}                                            
\usepackage{longtable}                                        
\usepackage{calc}                                             
\usepackage{multirow}                                         
\usepackage{hhline}                                           
\usepackage{ifthen}
\usepackage{caption} 
\captionsetup[table]{skip=3pt}  
\providecommand{\pr}[1]{\ensuremath{\Pr\left(#1\right)}}
\providecommand{\cbrak}[1]{\ensuremath{\left\{#1\right\}}}
\renewcommand{\thefigure}{\arabic{table}}
\renewcommand{\thetable}{\arabic{table}}                                     
                               
\title{AI1110 \\ Assignment 13}
\author{U.S.M.M TEJA \\ CS21BTECH11059}
\date{17th May 2022}


\begin{document}
	% The title page
	\begin{frame}
		\titlepage
	\end{frame}
	
	% The table of contents
	\begin{frame}{Outline}
    		\tableofcontents
	\end{frame}
	
	% The question
	\section{Question}
	\begin{frame}{question 15.25}
For the random walk model with two absorbing barriers , considering a slight generalization of the transition prabability matrix in (15-20) with $p_ij$ as in (15-19) for i\geq 1. In that case stated $e_0$ and $e_N $ atr absorbing states and $e_1$, $e_2$, ...., $e_N-1$ are transient states $f_0,0$ = 1, and $f_N,N$ = 1 and $f_N,0$ = 0
\end{frame}

\begin{frame}{theory}
 for j = 0
 
 \begin{align}
  & f_i,0 = q_if_i-1,0 + r_if_i,0 +p_if_i+1,0 for i\geq 1 &\\
  & or (f_i+1,0 - f_i,0)p_i = q_i (f_i1,0 - f_i-1,0)  & 
  \end{align}
Thus 
\begin{align}
   & f_i+1,0 - f_i,0 = \frac{q_i}{p_i}(f_i1,0 - f_i-1,0) =\frac{q_i.q_i-1 ... q_1}{p_i.p_i-1 ... p_1} (f_1,0 - 1) &\\
   & = \sigma_i(f_1,0 - 1)& \\
\end{align}
where
\end{frame}

    \begin{frame}{second Part-1}
where
\begin{align}
     & \sigma_i = \frac{q_i.q_i-1 ... q_1}{p_i.p_i-1 ... p_1} (f_1,0 - 1) & \\
 \end{align}
where,
\begin{align}
    & f_k,0 - 1 = \sum_{i=0}^k-1(f_i+1,0 - f_i,0) = \sum_{i=1}^k-1\sigma_i(f_1,0 - 1) &
\end{align}
With k = N we get $f_1,0$ = -1/\sum_{i=0}^N-1 $\sigma_i$ and hence starting from any transient state $e_k$ the desired probability of absorption into state $e_0$ is given by 
\begin{align}
    & f_k,0 = 1 - \frac{\sum_{i=0}^k-1 \sigma_i}{\sum_{i=0}^N-1 \sigma_i}  k=1,2,...N-1 given by & 
\end{align}

\end{frame}

\begin{frame}{second part-2}
  In  the special case of a uniform random walk $p_i$=p, $q_i$=q, $r_i$-0 so we have 
   \begin{align}
   &f_k,0 = 1 - \frac{1-(\frac{q}{p})^k}{1-(\frac{q}{p})^N} = \frac{(\frac{q}{p})^k - (\frac{q}{p})^N}{1 - (\frac{q}{p})^N}& \\
   & \frac{1-(\frac{p}{q})^N-k}{ - 1-(\frac{p}{q})^N}, k=1,2,3...N-1&
\end{align}

\end{frame}



\end{document}
